ss[a4paper]{article}
\usepackage[a4paper, total={7in, 10in}]{geometry}
\usepackage[english]{babel}
\usepackage[utf8]{inputenc}
\usepackage{amsmath}
\usepackage{graphicx}
\usepackage{listings}

\title{A comparison of programming paradigms}

\author{Devdeep Ray}

\date{\today}

\begin{document}
\maketitle

\begin{abstract}
Programming languages have evolved a lot over the last decade. This is an attempt 
to compare a select few and see where and when it is appropriate to choose a
language over the other possible choices. We focus on C, C++, DrRacket, Haskell 
and Prolog. We look at performance aspects, ease of development and clarity of 
code.
\end{abstract}

\section{Introduction}
As programs become larger and more involved, program design becomes very important.
Programs must be easy to understand and maintain so that it remains fresh. A lot of
people have spent a lot of time trying to fix bugs, make changes and pull up to
date lines and lines of old legacy code. Also, in this era of fast growing start-ups
and ever changing technology scene, development must be fast and efficient and it
should be easy to collaborate. Performance of code is also a concern which must be 
addressed. It is true that machines have become 100s of times faster than what they 
were a few decades ago, but it will still be economical if the code runs faster, 
because it would mean less hardware and less number of complicated things need 
to be done to make the program run at a larger scale. This makes choice of language
in a project a very critical decision. We look at a few programming languages that
span different paradigms and do a comparison of some aspects.

\section{Choice of languages for comparison}
We mainly look at C, C++, Haskell, DrRacket and Prolog. All of these languages support
separation of code into multiple files to aid modular programming and source code 
organization. All of these also have open source implementations, namly gcc for 
C and C++, ghc for haskell, raco for DrRacket and swi-prolog for Prolog.

\subsection{C} 
C is an imperative programming language. It is the baseline for performance
benchmarks. It is one of the fastest languages and the compiler is sufficiently
developed. It gives access to various low level semantics that can be used to
perform micro-optimizations, though the gcc compiler's automatic optimization 
techniques have made a lot of micro-optimizations obsolete. Basic data structures
provided by the language are the record data structure and arrays. It supports 
pointers to deal with references.

\subsection{C++}
C++ is also an imperative language along with various object oriented features. 
It has evolved from C, inheriting most features of C and adding various object 
oriented features into the language. C++ supports classes and constructors, 
and has a standard library of various containers which make it very convenient
to implement various algorithms. The containers include but are not limited 
to lists, dynamic arrays, stacks, queues, sets and maps. These libraries are
templated, which makes it very easy to use with different types of data types
and also makes it easy to write generic algorithms. It is important to see 
how using say vectors, instead of C arrays impacts the speed of a program. If
the impact is not huge, one can always use vectors instead of arrays because
of the convenience functions they provide. 

\subsection{Haskell}
Haskell is a strongly typed lazy functional programming language. This means
that there are no side effects in pure haskell. Haskell does support side 
effects through a language extension called monads. Monads are useful when 
we want to have side effects for performance or for performing IO. Monads 
are a very clever way of keeping the non functional part of the program 
separate and clearly visible. For example, a function which uses the IO 
monad, will have a return value which is in the IO monad. A pure function 
can be used to process a value in the IO monad, but the return value will 
also be tagged as IO monad. This is how using monads separates the pure part
from the impure part in haskell. Laziness in haskell helps in improving 
performance and it can also be exploited during programming. For example, it
is fairly straightforward to generate say, a list of all primes by applying 
a primality check to the infinite list of natural numbers. Since the language
is lazy, it will only do the primality check on the required number of 
natural numbers to get, say, the $i^{th}$ prime. It also caches evaluations 
so that it need not do them again and again. This is possible because of its
pure functional nature. Haskell is also a strongly typed language with an 
advanced type classing system. This makes the language less error prone because
type errors are caught during compile time. %% paper citation
Haskell can be interpreted or compiled. The ghc compiler can compile haskell 
to target code. ghci is a command line interpreter for haskell.

\subsection{DrRacket}
DrRacket is a multi-paradigm programming language that is derived from the
scheme/lisp family. It can be used to do functional programming with support
for higher order functions and the standard features of functional languages, 
and it also supports side effects and object oriented programming. It is a
weakly typed programming language, which means programmers can stop worrying 
about types. This can increase productivity and improve prototyping speed
in a significant number of cases. %%Paper citation
The racket compiler compiles programs to byte code to run on a JIT
compilation system.

\subsection{Prolog}
Prolog is a logic programming language. It is a declarative language. The 
program logic needs to be specified using facts and rules. It employs back 
tracking to solve problems. It is very useful in rule based AI and expert 
systems, and to solve certain types of problems that require backtracking, 
like puzzles. Prolog is a declarative language. 

\section{Programs}
\subsection{Topological sort}
In this first comparison, we write topological sort in C, C++, Haskell and 
DrRacket. Prolog is left out of the list because it it does not lend itself 
well to this, though it can be done. In all the cases, we make the assumption
that the graph nodes are indexed by an integer from 1 to N, where N is the 
number of nodes. This requirement is not necessary, but if we dont have this,
it adds a log(N) or a hash  multiplicative overhead to the O(N) algorithm if
we don't have pointers, because we need to look up the node from its id. In
all the programming languages, the idea is to do a depth first search, 
looking for back edges. If a back edge is found, the graph cannot have a 
sorted order, otherwise, we find the sorted order, which is the reverse 
order in which DFS finishes. 
\subsubsection{C++}
We structure the graph's nodes in the following way

\begin{lstlisting}[language=c++]
/**
 * Class representing a node in a directed graph.
 */
class GraphNode {
private:
  std::vector<GraphNode*> neighbors;
  int ind;
public:
  int n; // Placeholder for data to be stored in the node.

  GraphNode(int _ind) {
    ind = _ind;
  }

  GraphNode() {
    GraphNode(0);
  }

  void addNeighbor(GraphNode* node) {
    neighbors.push_back(node);
  }

  std::vector<GraphNode*> getNeighbors() {
    return neighbors;
  }
  
  int getInd() {
    return ind;
  }
};
\end{lstlisting}
This is a simple design that gives us the freedom of choosing how we
actually allocate the nodes in memory (contiguous or non contiguous).
Also, vectors were chosen because the setup of adjacency lists will be 
done once, and then we will just be accessing them. Hence, vectors are 
a better choice than lists. \verb+ind+ is the index of the node and it 
is assumed that each node has a unique value that lies between 0 and 
N-1, where N is the number of nodes. This helps us to keep track of 
nodes using a simple array instead of a hash or set, which makes the
program faster. This code is put into a header file \verb+graph_node.h+.
We now define the interface for the topological sort algorithm in a 
header \verb+topsort.h+. This looks like this
\begin{lstlisting}[language=c++]
#include <vector>
#include "graph_node.h"

/**
 * Takes in a vector of nodes in the graph and topologically sorts
 * the vector of nodes, returning a list of integers in the sorted order.
 * Returns empty list if topsort not possible.
 */
std::vector<GraphNode*> top_sort(const std::vector<GraphNode*> &graphNodes);
\end{lstlisting}
There are a couple of things to note here. Note that input is a reference 
to a vector of \verb+GraphNode*+ as opposed to a reference to a vector of 
\verb+GraphNode+. This again allows the code to be agnostic of where or how
the nodes were allocated. Also, note that we need to have something like 
returning an empty list if the graph could not be sorted. We could have
alternatively taken a boolean argument by reference and set it to true or
false, or some other arrangement like returning a pair, but both of these 
are quite ugly. Now let us look at the code for topological sort. This is
written in \verb+topsort.cpp+. 

\begin{lstlisting}[language=c++]
std::vector<GraphNode*> top_sort(const std::vector<GraphNode*> &graphNodes) {

  std::vector<GraphNode*> sorted;
  std::vector<bool> tempMarked(graphNodes.size(), false), seen(graphNodes.size(), false);
  for (auto node : graphNodes) {
    if (!explore(node, sorted, tempMarked, seen)) {
      return std::vector<GraphNode*>(); // Cycle was detected during DFS.
    }
  }

  reverse(sorted.begin(), sorted.end());
  return sorted;
}
\end{lstlisting}
The \verb+explore+ function does DFS on the node passed to it, and if it 
finds a cycle/back-edge, it returns false. It also populates the \verb+sorted+ 
vector as the DFS finishes. Lets look at the definition of \verb+explore+.

\begin{lstlisting}[language=c++]
bool explore(GraphNode* node, std::vector<GraphNode*> &sorted,
             std::vector<bool> &tempMarked, std::vector<bool> &seen) {
  int nodeInd = node->getInd();
  if (tempMarked[nodeInd]) {
    return false; // Cycle detected as we saw a back edge.
  }
  if (seen[nodeInd]) {
    return true; // No need to explore neighbors, already done.
  }

  tempMarked[nodeInd] = true; // Add a temporary mark.
  for (auto nbr : node->getNeighbors()) {
    if (!explore(nbr, sorted, tempMarked, seen)) {
      return false; // Cycle detected.
    }
  }

  tempMarked[nodeInd] = false; // Remove temporary mark.
  seen[nodeInd] = true; // Add permanent mark.
  sorted.push_back(node); // Add to sorted vector.
  return true;
}
\end{lstlisting}
That finishes the implementation. A simple analysis will show that the run 
time of the implementation is $O(E + V)$ where $E$ is the number of edges
and $V$ is the number of vertices. This is indeed the algorithm's run time as well. 

\subsection{C}
The C implementation is similar to the C++ implementation. The differences 
are that we use C arrays instead of vectors and we cannot use classes, only
structs. We still use bool instead of int variables (as in pure C), it just 
makes things cleaner. Here is \verb+graph_node.h+.
\begin{lstlisting}[language=c++]
struct GraphNode {
  int n; // Placeholder for data
  int ind;
  int numnbrs; // Number of neighbors
  GraphNode** nbrs;
};
\end{lstlisting}
Right from here, we start seeing differences between the C++ implementation
and C implementation. All members of the struct are by default, public and 
there is no access control. There are no constructors. We have to store the
number of neighbors. This is a common difference between using arrays and
vectors. Size information needs to be maintained manually for arrays allocated
on the heap. The \verb+sizeof+ operator is a compile time macro, and hence 
it will not work for heap allocations. Here is \verb+topsort.h+.
\begin{lstlisting}[language=c++]
#include "graph_node.h"

bool top_sort(int size, GraphNode** nodes, GraphNode** sorted);
\end{lstlisting}
Note the common pattern of having the size along with each array. This makes 
function signatures bigger. Also, note that since the sorted list is reference
parameter, we can and must return an indication of whether we could sort it 
or not. We choose to return true if it could be sorted, and false otherwise.
Here is the full implementation of \verb+top_sort+.
\begin{lstlisting}[language=c++]
bool top_sort(int size, GraphNode** nodes, GraphNode** sorted) {

  int curindex = 0; // Index of sorted node position
  bool *tempMarked = malloc(size * sizeof(bool));
  bool *seen = malloc(size * sizeof(bool));
  for (int i = 0; i < size; ++i) {
    tempMarked[i] = false;
    seen[i] = false;
  }
  for (int i = 0; i < size; ++i) {
    if (!explore(nodes[i], sorted, &curindex, tempMarked, seen)) {
      free(nodes); // Quick return needs freeing.
      free(sorted);
      return false;
    }
  }
  // Reverse the nodes.
  for (int i = 0; i < size / 2; ++i) {
	GraphNode* tmp = sorted[i];
	sorted[i] = sorted[size - i - 1];
	sorted[size - i - 1] = tmp;
  }
  free(nodes);
  free(sorted);
  return true;
}
\end{lstlisting}
One of the important consequences of programming in C where arrays are used frequently is 
the use of malloc for almost all array allocations. It is very rare that we know the size 
of the array during compile time. The programmer must be very careful to free all allocations. 
It is often that short circuiting is done by the programmer (for example, a quick return) 
and one should be careful about freeing resources at every such place. Finally, here is the
implementation of \verb+explore+.
\begin{lstlisting}[language=c++]
bool explore(GraphNode* node, GraphNode** sorted, int* index, bool* tempMarked, bool* seen) {
  int nodeInd = node->ind;
  if (tempMarked[nodeInd]) {
    return false; // Cycle detected as we saw a back edge.
  }
  if (seen[nodeInd]) {
    return true; // No need to explore neighbors, already done.
  }

  tempMarked[nodeInd] = true; // Add a temporary mark.
  
  GraphNode** nbrs = node->nbrs;
  for (int i = 0; i < node->numnbrs; ++i) {
    if (!explore(nbrs[i], sorted, index, tempMarked, seen)) {
      return false; // Cycle detected.
    }
  }

  tempMarked[nodeInd] = false;; // Remove temporary mark.
  seen[nodeInd] = true; // Add permanent mark.
  sorted[*index] = node; // Add to sorted vector.
  ++(*index);
  return true;
}
\end{lstlisting}
It is clear that although the C version does pretty much what the C++ version does, there
is some amount of code bloat and sacrifice of readability. What we want to know is 
whether it was worth it to do so in terms of performance of the code. 
\end{document}
